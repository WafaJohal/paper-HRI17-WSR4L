\documentclass{sig-alternate-05-2015}
\usepackage[utf8]{inputenc}

\begin{document}


\CopyrightYear{2017}
\setcopyright{rightsretained}
\conferenceinfo{HRI '17 Companion}{March 06-09, 2017, Vienna, Austria}
\isbn{978-1-4503-4885-0/17/03}
\doi{http://dx.doi.org/10.1145/3029798.3029801}


%\clubpenalty=10000
%\widowpenalty = 10000---

\title{Robots for Learning}


\numberofauthors{10} 
\author{
\alignauthor
Wafa Johal\\
       \affaddr{CHILI/LSRO Labs}\\
       \affaddr{École Polytechnique Fédérale Lausanne}\\
       \affaddr{Lausanne, Switzerland}\\
       \email{wafa.johal@epfl.ch}
% 2nd. author
\alignauthor
Paul Vogt\\
       \affaddr{Tilburg center for Cognition and Computation}\\
       \affaddr{Tilburg University}\\
       \affaddr{Netherlands}\\
       \email{p.a.vogt@uvt.nl}
% 3rd. author
\alignauthor James Kennedy\\
       \affaddr{Centre for Robotics and Neural Systems}\\
       \affaddr{Plymouth University}\\
       \affaddr{United-Kingdoms}\\
       \email{james.kennedy@plymouth.ac.uk}
\and  % use '\and' if you need 'another row' of author names
% 4th. author
Mirjam de Haas\\
\affaddr{Tilburg center for Cognition and Computation}\\
\affaddr{Tilburg University}\\
\affaddr{Netherlands}\\
\email{mirjam.dehaas@uvt.nll}
% 5th. author
\alignauthor Ana Paiva\\
       \affaddr{IST}\\
       \affaddr{University of Lisbon}\\
       \affaddr{Portugal}\\
       \email{ana.paiva@inesc-id.pt}
% 6th. author
\alignauthor Ginevra Castellano\\
       \affaddr{Uppsala University}\\
       \affaddr{Sweden}\\
       \email{ginevra.castellano@it.uu.se}
}
% There's nothing stopping you putting the seventh, eighth, etc.
% author on the opening page (as the 'third row') but we ask,
% for aesthetic reasons that you place these 'additional authors'
% in the \additional authors block, viz.
\additionalauthors{Additional authors: 
Sandra Okita (Teachers College - Columbia University, United States, email: {\texttt{Okita@exchange.tc.columbia.edu}}),
 Fumihide Tanaka
(University of Tsukuba, Japan, email: {\texttt{tanaka@iit.tsukuba.ac.jp}}), 
Tony Belpaeme (Centre for Robotics and Neural Systems, Plymouth University, U.K. and Ghent University, email: {\texttt{tony.belpaeme@plymouth.ac.uk}}) and
Pierre Dillenbourg (CHILI Lab, École Fédérale Polytechnique Lausanne, Switzerland, email: {\texttt{pierre.dillenbourg@epfl.ch}}). 
}
%\date{30 July 1999}
% Just remember to make sure that the TOTAL number of authors
% is the number that will appear on the first page PLUS the
% number that will appear in the \additionalauthors section.

\maketitle
\begin{abstract}
An increasing amount of HRI research focuses on the development of social robots acting as tutors. While robots have been popular as a focus for STEM teaching (see Lego Mindstorms or Thymio), the use of robots as tutors is novel. The field of HRI has started reporting on how to make effective robot tutors and how to measure their efficacy. These studies have shown that the potential of robots in educational settings is inarguable: robot can provide educational content tailored to the individual, something which is missing from current educational settings. They also have the potential to enhance learning via kinesthetic interaction, can improve the learner’s self-esteem and can provide empathic feedback. Finally, robots have been shown to engage the learner, to motivate her in the learning task or to enhance collaboration in a group. 
However, many questions still remain. For instance, what interaction strategies aid learning, and which hamper learning? How can we deal with the current technical limitations of robots? How should effective lessons be developed and implemented on a robot? Answering these and other questions requires a multidisciplinary effort, including contributions from pedagogy, developmental psychology, (computational) linguistics, artificial intelligence and HRI, among others. 
The aim of this workshop is to engage scholars who aim to gain expertise in education and in robotics (from instructional design to inverse kinematics, ROS to ZPD, Markov to Piaget) into a new  interdisciplinary community working on educational robotics. Participants will benefit from hearing from the forefront of field and from discussions on how to move from fundamental research towards the development of market-ready educational robots.
\end{abstract}


%
% The code below should be generated by the tool at
% http://dl.acm.org/ccs.cfm
% Please copy and paste the code instead of the example below. 
%
 \begin{CCSXML}
	<ccs2012>
	<concept>
	<concept_id>10010520.10010553.10010554.10010558</concept_id>
	<concept_desc>Computer systems organization~External interfaces for robotics</concept_desc>
	<concept_significance>500</concept_significance>
	</concept>
	<concept>
	<concept_id>10010405.10010489</concept_id>
	<concept_desc>Applied computing~Education</concept_desc>
	<concept_significance>500</concept_significance>
	</concept>
	<concept>
	<concept_id>10003120.10003121</concept_id>
	<concept_desc>Human-centered computing~Human computer interaction (HCI)</concept_desc>
	<concept_significance>500</concept_significance>
	</concept>
	</ccs2012>
\end{CCSXML}

\ccsdesc[500]{Computer systems organization~External interfaces for robotics}
\ccsdesc[500]{Applied computing~Education}
\ccsdesc[100]{Human-centered computing~Human computer interaction (HCI)}

%
% End generated code
%

%
%  Use this command to print the description
%
\printccsdesc

\keywords{Human-Robot Interaction, Robots in Education, Tutor Robots, Child-Robot Interaction}

%\section{Introduction}


%ACKNOWLEDGMENTS are optional
\section{Acknowledgments}
We would like to thank the Swiss National Science Foundation href{http://www.nccr-robotics.ch/}{National Centre of Competence in Research Robotics}, \dots

\bibliographystyle{abbrv}
\bibliography{hriwk10_r4l}  

\end{document}

\documentclass{sig-alternate-05-2015}
\usepackage[bookmarks=true]{hyperref}
\usepackage[utf8x]{inputenc}
\usepackage{todonotes}
\usepackage{enumitem}

\newcommand{\comments}[1]{\todo[inline,linecolor=blue,backgroundcolor=blue!15,bordercolor=blue]{#1}}


\begin{document}

\CopyrightYear{2017}
\setcopyright{rightsretained}
\conferenceinfo{HRI '17 Companion}{March 06-09, 2017, Vienna, Austria}
\isbn{978-1-4503-4885-0/17/03}
\doi{http://dx.doi.org/10.1145/3029798.3029801}

%\clubpenalty=10000
%\widowpenalty = 10000---

\title{Workshop on Robots for Learning -  R4L}

\numberofauthors{10} 
\author{
\alignauthor
Wafa Johal\\
       \affaddr{CHILI/LSRO Labs}\\
       \affaddr{{\'E}cole Polytechnique}\\
       \affaddr{F{\'e}d{\'e}rale Lausanne}\\
       \affaddr{Lausanne, Switzerland}\\
       \affaddr{wafa.johal@epfl.ch}
% 2nd. author
\alignauthor
Paul Vogt\\
       \affaddr{Tilburg center for Cognition and Computation}\\
       \affaddr{Tilburg University}\\
       \affaddr{Netherlands}\\
       \affaddr{p.a.vogt@uvt.nl}
% 3rd. author
\alignauthor James Kennedy\\
       \affaddr{Centre for Robotics and Neural Systems}\\
       \affaddr{Plymouth University}\\
       \affaddr{United Kingdom}\\
       \affaddr{james.kennedy\\@plymouth.ac.uk}
\and  % use '\and' if you need 'another row' of author names
% 4th. author
\alignauthor Mirjam de Haas\\
	   \affaddr{Tilburg center for Cognition}\\
	   \affaddr{and Computation}\\
	   \affaddr{Tilburg University}\\
	   \affaddr{Netherlands}\\
	   \affaddr{mirjam.dehaas@uvt.nl}
% 5th. author
\alignauthor Ana Paiva\\
       \affaddr{Instituto Superior T{\'e}cnico}\\
       \affaddr{University of Lisbon}\\
       \affaddr{Portugal}\\
       \affaddr{ana.paiva@inesc-id.pt}
% 6th. author
\alignauthor Ginevra Castellano\\
	   \affaddr{Department of Information Technology}\\
       \affaddr{Uppsala University}\\
       \affaddr{Sweden}\\
       \affaddr{ginevra.castellano\\@it.uu.se}
}
% There's nothing stopping you putting the seventh, eighth, etc.
% author on the opening page (as the 'third row') but we ask,
% for aesthetic reasons that you place these 'additional authors'
% in the \additional authors block, viz.
\additionalauthors{
\begin{itemize}[noitemsep,nolistsep,leftmargin=.1in]
\item Sandra Okita, Teachers College - Columbia University, United States, {\texttt{okita@tc.columbia.edu}}
\item Fumihide Tanaka, University of Tsukuba, Japan, \\ {\texttt{tanaka@iit.tsukuba.ac.jp}}
\item Tony Belpaeme, Centre for Robotics and Neural Systems, Plymouth University, U.K. and Ghent University, Belgium, {\texttt{tony.belpaeme@plymouth.ac.uk}}
\item Pierre Dillenbourg, CHILI Lab, {\'E}cole F{\'e}d{\'e}rale Polytechnique Lausanne, Switzerland, {\texttt{pierre.dillenbourg@epfl.ch}}
\end{itemize}
}
%\date{30 July 1999}
% Just remember to make sure that the TOTAL number of authors
% is the number that will appear on the first page PLUS the
% number that will appear in the \additionalauthors section.

%TODO fix author emails going outside page boundary

\maketitle

%-----------------------------------------------------------------------------
\begin{abstract}
 While robots have been popular as a tool for 
STEM teaching, the use of robots in other learning scenario is novel. 
The field of HRI has 
started to report on how to make effective robots usable in educational contexts. However, many 
challenges remain. For instance, what interaction strategies aid learning, and 
which hamper learning? How can we deal with the current technical limitations of 
robots? Answering these and other questions requires a multidisciplinary effort, 
including contributions from pedagogy, developmental psychology, (computational) 
linguistics, artificial intelligence and HRI, among others. This abstract 
provides an overview of the current state-of-the-art in robots designed for learning with social aspects and 
describes the aims of the Robots for Learning (R4L) workshop in bringing 
together a multidisciplinary audience for furthering the development of 
market-ready educational robots.
\end{abstract}
%-----------------------------------------------------------------------------

%TODO JK: do we need to include this? It wasn't required on the paper for full papers so removed for now
%
% The code below should be generated by the tool at
% http://dl.acm.org/ccs.cfm
% Please copy and paste the code instead of the example below. 
%
\begin{CCSXML}
	<ccs2012>
	<concept>
	<concept_id>10010520.10010553.10010554.10010558</concept_id>
	<concept_desc>Computer systems organization~External interfaces for robotics</concept_desc>
	<concept_significance>500</concept_significance>
	</concept>
	<concept>
	<concept_id>10010405.10010489</concept_id>
	<concept_desc>Applied computing~Education</concept_desc>
	<concept_significance>500</concept_significance>
	</concept>
	<concept>
	<concept_id>10003120.10003121</concept_id>
	<concept_desc>Human-centered computing~Human computer interaction (HCI)</concept_desc>
	<concept_significance>500</concept_significance>
	</concept>
	</ccs2012>
\end{CCSXML}

\ccsdesc[500]{Computer systems organization~External interfaces for robotics}
\ccsdesc[500]{Applied computing~Education}
\ccsdesc[100]{Human-centered computing~Human computer interaction (HCI)}

%
% End generated code
%

%
%  Use this command to print the description
%
%\printccsdesc

\keywords{Human-Robot Interaction, Robots in Education, Tutor Robots, Child-Robot Interaction}

%-----------------------------------------------------------------------------
\section{Introduction}
%-----------------------------------------------------------------------------
% 1 paragraph high-level overview
%TODO refs throughout this par
%TODO \comments{the jump to social robotics was a bit fast, I did some modicfications to include any robots using for educational purposes that aimto increase social interaction either between the robot and the user or among users} 
An increasing amount of HRI research is focused on the development of applications of service robots in everyday life. 
In education, while robots have been popular as a focus for STEM 
teaching (cf. Lego Mindstorms or Thymio \cite{riedo2012two}), the use of robots in other learning scenarios is novel. The field of HRI has started reporting on how to make 
effective robots and how to measure their efficacy 
\cite{kennedy2016social,tanaka2015pepper}. Robots have the 
potential to enhance learning via kinesthetic interaction \cite{lemaignan2016,tanaka2012children}, can improve 
the learner's self-esteem \cite{lemaignan2016}, and can provide empathic feedback 
\cite{castellano2013towards}. Finally, robots have been shown to engage the 
learner, to motivate her in the learning task or to stimulate collaboration in a 
group \cite{}. However, many challenges remain and this workshop aims to bring 
together a multidisciplinary group of researchers to discuss these challenges 
and share expertise.

% 1 paragraph talk about previous version of workshop (what was done there, and what this will add)
The second iteration of this workshop builds on the previous version hosted at 
the IEEE International Symposium on Robot and Human Interactive Communication 
(RO-MAN), 2016. The previous workshop utilised keynote speakers, participant 
speakers, and small group discussions to raise issues and challenges facing the 
community researching robots for use in delivering educational content. The 
second version of this workshop seeks to engage with more researchers in the 
field, and draw a more multidisciplinary audience to further the development of 
market-ready educational robots.

%-----------------------------------------------------------------------------
\section{Background}
%-----------------------------------------------------------------------------

% overview state-of-the-art for robots for learning (maybe also clarify what is meant by the term)

Mubin \cite{mubin2013review} distinguishes three roles for robots in education:
1) \textbf{tutors} - providing help to students, 2) \textbf{peers} - stimulating learning, and 3) \textbf{tools} - physically enhancing a concept to learn.

At first, in the 70's and 80's robots tended to be introduced in schools as a tool for teachers to teach about robotics or other STEM subjects. 
However, this specificity of robot use penalized their adoption in educational contexts \cite{}. 
Nowadays, with robots being cheaper and more easily deployable, application in education becomes possible for other types of learning.  


The current state of the art presents robots used in various learning scenarios related to non-programming curricula. 
Often involving social robots these scenario usually investigate the social aspect of the robot-learner relationship: empathy \cite{castellano2013towards}, immediacy, spatial arrangement\cite{}, \dots
While certain research focus in one-to-one setup \cite{} exploiting social and task adaptive systems to individuals, others aim to provide a tool for the therapist or educators in  their teaching practice\cite{}. 

limits of the soa:
- Design of this experiment often without involving educational specialist or practitioners. 
- long term use of robots
- many works have been done in CSCL to introduce technologies (computers, tablets and other hci device ias tools) need to llok at how and why
- proving the impact of learning with a robot, is it only motivational?

% close by highlighting some challenges (which we can talk about addressing in outline...)
%However, many questions still remain. For instance, what interaction strategies aid learning, and which hamper learning? How can we deal with the current technical limitations of robots? How should effective lessons be developed and implemented on a robot? Answering these and other questions requires a multidisciplinary effort, including contributions from pedagogy, developmental psychology, (computational) linguistics, artificial intelligence and HRI, among others. 

%-----------------------------------------------------------------------------
\section{Outline of the Workshop}
%-----------------------------------------------------------------------------
% who is the workshop intended for?
The aim of this workshop is to engage scholars who wish to gain expertise in 
education and in robotics. Participants will benefit from hearing from the 
forefront of field and from discussions on how to move from fundamental research 
towards the development of market-ready educational robots.

% how will the aims be achieved on the day?
The workshop aims will be achieved through presentations and discussions. 
Prospective participants are invited to submit 4-6 page papers describing work in progress, or containing preliminary results to discuss with the community.
In order to stimulate interactions, the workshop will include short position paper presentations (10+2min) and poster sessions. 
The afternoon will be dedicated to discussion, including both a panel session and semi-structured group discussions.
The 

%-------------------------------------------
\section{Organizers}
% some workshop abstracts include details about the organizers...but maybe we should avoid that given that it would produce a short novel.
Wafa Johal, PhD. is a postdoctoral researcher within the CoWriter and Cellulo projects dealing with robots for education in the CHILI and LSRO Labs at EPFL. She obtained her PhD in 2015 from the University of Grenoble (France) focusing on body signals in Child-Robot Interaction.  

Paul Vogt is Associate Professor in Language learning and HRI. He is a trained cognitive scientist and holds a PhD in Artificial Intelligence. His research focuses on first and second language acquisition using methods ranging from ethnographic research and psycholinguistics to computational modelling of language acquisition and HRI. Paul is one of the principal investigators in the L2TOR project.

James Kennedy is currently completing his PhD in Human-Robot Interaction at Plymouth University (U.K.). His research interests centre around social companion robots, particularly in educational interactions with children. He has been involved with the ALIZ-E, DREAM and L2TOR European projects.

Mirjam de Haas finished her master in Artificial Intelligence and is a PhD student in the L2TOR project. Her research focuses on the interaction between robots and children and how to design a child-friendly robot.

Ana Maria Paiva's main scientific interests lay in the area of Autonomous Agents, Embodied Conversational Agents and Robots and Multiagent Simulation Systems. She has been researching in the area of artificial intelligence for the past twenty years. She is the principal investigator of the eCUTE aiming in exploring technologically-enhanced learning approaches for intercultural understanding.

Ginevra Castellano is an associate senior lecturer in intelligent interactive systems at Uppsala University, where she leads the Social Robotics Lab. She was the coordinator of the EMOTE project (2012-2016), which developed educational robots to support teachers in a classroom environment.

Sandra Okita is an Associate Professor of Technology and Education at Teachers College, Columbia University. Her current research interest is focused on the learning partnership between individuals and technology, and how technology intersects with learning and instructional processes.

Fumihide Tanaka, PhD,  has been actively working in the area of educational robots and child-robot interaction, and now is recognized as one of the pioneers in this research area. He moved to academia in 2008, the University of Tokyo (2014), and is now at the University of Tsukuba.

Tony Belpaeme's research focuses on cognitive robotics and social Human-Robot Interaction, in which natural and artificial cognition is considered to be closely intertwined with social interaction. He coordinates the L2TOR project on learning a second language using robot tutors, and collaborates on several international research projects on HRI and cognitive robotics.

Pierre Dillenbourg is a former teacher in elementary school. He graduated in educational science (University of Mons, Belgium). His research on learning technologies started in 1984. He obtained a PhD in computer science from the University of Lancaster (UK), in artificial intelligence applications for educational software. He is currently full professor in learning technologies, head of the CHILI Lab involved in both CoWriter and Cellulo projects.

%-----------------------------------------------------------------------------
\section{Acknowledgments}
We would like to thank the Swiss National Science Foundation 
\href{http://www.nccr-robotics.ch/}{National Centre of Competence in Research 
Robotics}, the EU H2020 L2TOR project (grant no. 688014),\dots

%-----------------------------------------------------------------------------
\bibliographystyle{abbrv}
\bibliography{hriwk10_r4l}  

\end{document}
